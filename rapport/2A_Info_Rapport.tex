\documentclass[a4paper,11pt]{myreport}
%\documentclass{scrreprt}

\usepackage[dvipsnames]{xcolor}
\definecolor{souris}{gray}{0.19}
\definecolor{tagada}{RGB}{35,0,35}
\colorlet{bordeaux}{red!75!blue!25!darkgray}
\usepackage[hyperfootnotes=false]{hyperref}
\usepackage[toc,page]{appendix} 
\usepackage[utf8]{inputenc}
\usepackage[T1]{fontenc}
\usepackage{geometry}
\usepackage[export]{adjustbox}
\geometry{left=75pt,right=75pt,bottom=75pt}
\usepackage{a4wide}
\usepackage[francais]{babel}
\usepackage[babel=true]{csquotes} % csquotes va utiliser la langue définie dans babel
\usepackage{graphics}
\usepackage{movie15}
\usepackage{graphicx}
\usepackage{verbatim}
\usepackage{listings}
\usepackage[capbesideposition=bottom]{floatrow}
%\floatsetup{capposition=bottom}
\usepackage{caption}
\usepackage{color}
\usepackage{fancyhdr}
\usepackage[bottom]{footmisc}


\pagestyle{fancy}
\renewcommand{\footrulewidth}{1pt}
\fancyfoot[C]{\textbf{page \thepage}} 
\fancyfoot[L]{Projet 2A Informatique}
\fancyfoot[R]{Emmanuel Breton\--\--Belz}
\renewcommand{\footrulewidth}{0.7pt}
%\usepackage{syntonly}
%\newenvironment{DocStage}
%\syntaxonly
%%\setlength{\oddsidemargin}{dim souhaitée}
%%\setlength{\evensidemargin}{dim souhaitée}
%%\setlength{\textwidth}{dim souhaitée}
%%\setlength{\headheight}{dim souhaitée}
%%\setlength{\topmargin}}{dim souhaitée}
%%\setlength{\footskip}{90pt}

\setcounter{tocdepth}{2}
 \hypersetup{	
colorlinks=true, %colorise les liens 
breaklinks=true, %permet le retour à la ligne dans les liens trop longs 
urlcolor= blue, %couleur des hyperliens 
linkcolor= tagada ,	%couleur des liens internes 
citecolor=back,	%couleur des références 
pdftitle={Rapport de projet 2A}, %informations apparaissant dans 
pdfauthor={Emmanuel Breton\--\--Belz}, %les informations du document 
pdfsubject={Projet 2A}	%sous Acrobat. 
} 
\title{Jeu vidéo pédagogique utilisant la technologie Oculus VR}
\author{Emmanuel Breton\--\--Belz}


\begin{document} % ------------------------begin---------------------------------------
\def\siecle#1{\textsc{\romannumeral #1}\textsuperscript{e}~si\`ecle}

%reglages pour l’insertion de code
\lstset{language=Java,
basicstyle=\normalsize, % ou ça==> basicstyle=\scriptsize,upquote=true,
aboveskip={1.5\baselineskip},
columns=fullflexible,
showstringspaces=false,
extendedchars=true,
breaklines=true,
showtabs=false,
showspaces=false,
showstringspaces=false,
identifierstyle=\ttfamily,
keywordstyle=\color[rgb]{0,0,1},
commentstyle=\color[rgb]{0.133,0.545,0.133},
stringstyle=\color[rgb]{0.627,0.126,0.941},
}
\nopagebreak[3]


%\maketitle -------------------------------- premiere page
%%\setlength{\textheight}{28cm}
%%\setlength{\topmargin}{-5pt}

\makeatletter

  \begin{titlepage}
  \centering
      {\large \textsc{école nationale supérieur d\rq{}ingénieurs de caen}}\\
      \textsc{Dans le cadre du projet 2A}\\
    \vspace{1cm}
    \vspace{1cm}
      {\large\textbf{	\@date\\
      \LARGE{Rapport projet de 2\up{ème} année}}}
      
    \vfill
      \includegraphics[width=0.25\textheight,center]{./images/LogoEnsicaenSansTexte.jpg}
    \vfill
       {\LARGE \textbf{\@title}} \\
    \vspace{1em}
        {\large \@author} \\
    \vfill

        \includegraphics[width=0.25\textheight,left]{./images/oculus_logo.jpg}
        \includegraphics[width=0.25\textheight,right]{./images/Unity_3D_logo.png}

  \end{titlepage}
\makeatother
%\end --------------------------------------------------------------------

%table des matières :
\setlength{\textheight}{26cm}
\setlength{\topmargin}{-2cm}

\tableofcontents
\section{Remerciements}

\listoffigures
\chapter{Introduction}

\chapter{Contexte}

\chapter{Réalisation}
\section{Outils}
\section{Composants}
\chapter{Difficultés}

\chapter{Résultats}

\chapter{Glossaire}

\chapter{Annexes}

\begin{figure}[h]
	\includegraphics[scale=0.70]{./images/LogoEnsicaenSansTexte.jpg}
	\caption{JDialog permettant le réglage d'une LED 4 couleurs}
\end{figure}

\end {document}
